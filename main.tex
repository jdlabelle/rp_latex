% Document class
\documentclass[12pt, titlepage]{article}

% Required packages
\usepackage[letterpaper, margin=1in]{geometry}
\usepackage{times}  % Times New Roman font
\usepackage{setspace}  % For double spacing
\usepackage{parskip}  % For paragraph spacing
\usepackage{fancyhdr}  % For headers and footers
\usepackage{cite}  % For citations
\usepackage{apacite}  % For APA citation style
\usepackage{indentfirst}  % For paragraph indentation
\usepackage{url} % For URL usage

% Document settings
\doublespacing  % Double spacing for main content
\setlength{\parindent}{0.5in}  % Paragraph indentation

% Header and footer settings
\pagestyle{fancy}
\fancyhf{}
\fancyfoot[C]{\thepage}  % Page numbers in center of footer
\renewcommand{\headrulewidth}{0pt}  % Remove header line
\renewcommand{\footrulewidth}{0.4pt}  % Footer line thickness

% Remove all section numbering:
% \setcounter{secnumdepth}{0}

% Title page settings
\title{\vspace{-1.5in}Your Paper Title}
\author{Your Name\\Your Institution}
\date{\today}

\begin{document}

% Title page
\maketitle

% Abstract (optional)
\begin{abstract}
  Your abstract goes here. Keep it concise and focused on the main points of your 
  research. The abstract should be a single paragraph that summarizes your key 
  findings and conclusions.
\end{abstract}

% Introduction
\section{Introduction}
%\section*{Introduction}  -- remove numbering for this section
% If you use \section*{}, those sections won't appear in an automatically generated 
% Table of Contents (ToC) unless you explicitly add them with \addcontentsline.
  The introduction should provide context for your research and outline the main 
  objectives. Start with a broad overview of the topic and gradually narrow down 
  to your specific research question. You can cite sources using 
  \cite{example2024} to support your statements.

  Each paragraph should focus on a single main idea. When you need to start a new 
  paragraph, leave a blank line in the source file. This makes the structure of 
  your document clear and easy to edit.

% Main content sections
\section{Methodology}
  Describe your research methods in detail. This section should be organized 
  logically, explaining your approach step by step. Use subsections when 
  appropriate:

  \subsection{Data Collection}
    Explain how you gathered your data. Be specific about your procedures and 
    justify your choices.

  \subsection{Analysis}
    Describe your analysis methods and any statistical procedures used.

\section{Results}
  Present your findings clearly and systematically. Use tables and figures when 
  appropriate, and refer to them in your text:

  \begin{table}[ht]
    \centering
    \begin{tabular}{lcc}
      \hline
      Category & Count & Percentage \\
      \hline
      A & 10 & 20\% \\
      B & 15 & 30\% \\
      C & 25 & 50\% \\
      \hline
    \end{tabular}
    \caption{Example table showing data distribution}
    \label{tab:distribution}
  \end{table}

\section{Discussion}
  Interpret your results and discuss their implications. Connect your findings 
  back to the literature you reviewed in the introduction. Address any 
  limitations of your study and suggest directions for future research.

% Conclusion
\section{Conclusion}
  Summarize your main findings and their significance. Keep this section concise 
  and focused on the key takeaways from your research.

% Bibliography
\clearpage % Bibliography on its own page
\singlespacing  % Single spacing for bibliography
\bibliographystyle{apacite}
\bibliography{references}

\end{document} 
